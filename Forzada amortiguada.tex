\documentclass[a4paper,12pt]{article}
\usepackage[utf8]{inputenc}
\usepackage{amsmath, amssymb}
\usepackage{graphicx}
\usepackage{float}
\usepackage{geometry}
\geometry{margin=2.5cm}
\usepackage{hyperref}

\title{Informe de Laboratorio: Vibraciones Mecánicas\\
Sección: Forzada Amortiguada}
\author{Nombre: \\ Matrícula: \\ Grupo: }
\date{\today}

\begin{document}

\maketitle

\newpage

% --- INICIO DEL INFORME ---
% =========================================================
% SECCIÓN: Vibración Forzada-Amortiguada
% (Experimento 4)
% =========================================================

\section{Vibración forzada-amortiguada}

%----------------------------------------------------------
\subsection{Modelo teórico}
% Aquí debes escribir la deducción del modelo para el banco experimental,
% adaptando del modelo masa-resorte-amortiguador al sistema rotacional.

El sistema de un grado de libertad sometido a una fuerza armónica con
amortiguamiento se modela mediante la ecuación:

\begin{equation}
    m\,\ddot{x}(t) + c\,\dot{x}(t) + k\,x(t) = P_m \cos(\omega_f t)
    \label{eq:forzada_amortiguada_lineal}
\end{equation}

% Si quieres escribir el modelo rotacional, puedes cambiar m,k por I,k_\theta, etc.
% y x(t) por \theta(t).

La solución en régimen permanente se escribe como:

\begin{equation}
    x(t) = X \cos(\omega_f t - \varphi)
\end{equation}

donde la amplitud de respuesta es:

\begin{equation}
    X = \dfrac{P_m}{\sqrt{(k - m \omega_f^2)^2 + (c \,\omega_f)^2}}
    \label{eq:amplitud_forzada_amortiguada}
\end{equation}

y el desfase entre la fuerza y la respuesta es:

\begin{equation}
    \tan \varphi = 
    \dfrac{2\,\zeta\,r}{1 - r^2}
    \qquad
    \text{con} \quad
    r = \dfrac{\omega_f}{\omega_n}, \quad
    \zeta = \dfrac{c}{C_{\text{crítico}}}
    \label{eq:desfase_teorico}
\end{equation}

El factor de amplificación se define como:

\begin{equation}
    FA = \dfrac{X}{X_{\text{estático}}}
       = \dfrac{1}{\sqrt{(1 - r^2)^2 + (2 \zeta r)^2}}
    \label{eq:FA_teorico}
\end{equation}

La fuerza de excitación debida a la masa excéntrica se calcula como:

\begin{equation}
    P_m = M_h\, r\, \omega_f^2
    \label{eq:F_exc}
\end{equation}

donde $M_h$ es la masa equivalente del orificio, $r$ es el radio de excentricidad
y $\omega_f$ la velocidad angular de las masas excéntricas.

% (Opcional) Aquí puedes escribir la versión ROTACIONAL del modelo,
% usando el momento de inercia total I, el brazo L, etc.
%----------------------------------------------------------

\subsection{Descripción del montaje experimental}

% Explica brevemente que es el mismo sistema de los experimentos anteriores
% pero con amortiguador instalado en configuraciones abierta y cerrada.

El montaje corresponde al sistema del banco HVT12 con:

\begin{itemize}
    \item Viga rígida apoyada en un pivote.
    \item Resorte lineal con constante elástica $k = [\dots]\ \text{kN/m}$.
    \item Motor con masas excéntricas HAC120.
    \item Sistema de amortiguamiento con discos en configuración:
          \textbf{abierta} y \textbf{cerrada}.
    \item Sensor LVDT para medir el desplazamiento del extremo de la viga
          y convertirlo en ángulo de oscilación $\theta$.
\end{itemize}

%----------------------------------------------------------

\subsection{Procedimiento experimental}

% Aquí solo dejas el esquema; luego completas detalles (valores, etc.).

\begin{enumerate}
    \item Configurar el amortiguador en la posición \textbf{abierta}.
    \item Encender el motor y seleccionar una frecuencia $f$ cercana a la
          frecuencia natural $f_n$ (diferencia máxima de $0.3\,$Hz).
    \item Esperar a que el sistema alcance régimen permanente.
    \item Registrar simultáneamente:
          \begin{itemize}
              \item Señal del LVDT (desplazamiento / ángulo).
              \item Señal del sensor de proximidad (referencia de fase).
          \end{itemize}
    \item Repetir el procedimiento para al menos tres frecuencias diferentes
          en configuración \textbf{abierta}.
    \item Repetir los pasos anteriores para la configuración \textbf{cerrada}
          del amortiguador.
    \item Exportar los datos en formato \texttt{.csv} para su análisis.
\end{enumerate}

%----------------------------------------------------------

\subsection{Procesamiento de datos}

% En esta subsección describes qué cálculos vas a hacer con los datos medidos.
% Deja aquí las fórmulas y luego rellenas con tus resultados reales.

\begin{enumerate}
    \item Convertir las frecuencias forzadas $f$ a frecuencia angular:
    \begin{equation}
        \omega_f = 2\pi f
    \end{equation}

    \item Calcular la relación de frecuencias:
    \begin{equation}
        r = \dfrac{\omega_f}{\omega_n}
    \end{equation}

    \item Calcular la fuerza de excitación usando la ecuación
    \eqref{eq:F_exc} con:
    \[
        M_h = 56.16\ \text{g}, \qquad r = 45\ \text{mm}
        % (puedes ponerlos en SI en tu informe)
    \]

    \item Obtener el ángulo máximo experimental $\theta^{\text{exp}}_m$
          a partir de la señal del LVDT:
          % Explica brevemente cómo lo haces (pico a pico, FFT, etc.)
          \begin{equation}
              \theta^{\text{exp}}_m = \arctan\left(\dfrac{\delta_{\max}}{d}\right)
          \end{equation}
          donde $\delta_{\max}$ es el desplazamiento máximo y
          $d$ la distancia del LVDT al pivote.

    \item Calcular la amplitud teórica $\theta^{\text{teo}}_m$ a partir de
          la ecuación \eqref{eq:amplitud_forzada_amortiguada} y la
          correspondiente relación entre desplazamiento y ángulo.

    \item Calcular el factor de amplificación:
    \begin{equation}
        FA = \dfrac{\theta_m}{\theta_{\text{estática}}}
    \end{equation}
    % Aquí puedes especificar cómo obtienes \theta_estática.

    \item Calcular el desfase teórico usando \eqref{eq:desfase_teorico}.

    \item Calcular el desfase experimental $\varphi_{\text{exp}}$ usando
          la diferencia de tiempo $\Delta t$ entre la señal del LVDT y la
          del sensor de proximidad:

    \begin{equation}
        \varphi_{\text{exp}} =
        -\left(\dfrac{\Delta t}{T}\right) 2\pi - \dfrac{\pi}{2}
        \label{eq:desfase_exp}
    \end{equation}

    donde $T$ es el período de la excitación.
\end{enumerate}

%----------------------------------------------------------

\subsection{Resultados}

% Aquí van las tablas y figuras. Te dejo la estructura lista.

\subsubsection*{Tabla de resultados}

\begin{table}[h]
    \centering
    \caption{Resultados experimentales y teóricos para vibración forzada-amortiguada}
    \label{tab:forzada_amortiguada}
    \begin{tabular}{lcccccccc}
        \hline
        Caso & $f$ [Hz] & $P_m$ [N] & $\theta^{\text{exp}}_m$ [°] &
        $\theta^{\text{teo}}_m$ [°] & $FA$ &
        $r = \omega_f / \omega_n$ &
        $\varphi_{\text{teo}}$ [°] & $\varphi_{\text{exp}}$ [°] \\
        \hline
        Abierta & 4.2 & 1.7599 & 0.6878 & 7.1843 & 1.7895 & 0.6642 & 0.3746 & -2.3020 \\
        Abierta & 4.3 & 1.8447 & 0.6878 & 7.1843 & 1.8603 & 0.6800 & 0.3987 & -4.5617 \\
        Abierta & 4.45 & 1.9757 & 0.6878 & 7.1843 & 1.9813 & 0.7038 & 0.4394 & -3.4542 \\
        Cerrada & 4.2 & 1.7599 & 0.6878 & 7.1843 & 1.7896 & 0.6642 & 0.1182 & -4.5506 \\
        Cerrada & 4.3 & 1.8447 & 0.6878 & 7.1843 & 1.8603 & 0.6800 & 0.1258 & -4.0398 \\
        Cerrada & 4.45 & 1.9757 & 0.6878 & 7.1843 & 1.9814 & 0.7038 & 0.1387 & -3.3399 \\
        \hline
    \end{tabular}
\end{table}

\subsubsection*{Datos adicionales}
\begin{table}[h]
    \centering
    \caption{Datos adicionales del sistema}
    \label{tab:datos_adicionales}
    \begin{tabular}{cc}
        \hline
        $\omega_f$ & $\omega_n$ & $\zeta$ \\
        \hline
        26.3894 & 39.729 & 0.00275~(abierta) \\
        27.0177 & 39.729 & 0.00275~(abierta) \\
        27.9602 & 39.729 & 0.00275~(abierta) \\
        26.3894 & 39.729 & 0.00086805~(cerrada) \\
        27.0177 & 39.729 & 0.00086805~(cerrada) \\
        27.9602 & 39.729 & 0.00086805~(cerrada) \\
        \hline
    \end{tabular}
\end{table}

\subsubsection*{Gráficas}

\begin{figure}[h]
    \centering
    %\includegraphics[width=0.8\textwidth]{FA_vs_r_abierta_cerrada.pdf}
    \caption{Curvas de factor de amplificación $FA$ en función de
             $\omega_f/\omega_n$ para las configuraciones abierta y cerrada.
             % ---> Aquí describe brevemente qué curva corresponde a qué configuración.
    }
    \label{fig:FA_vs_r}
\end{figure}

\begin{figure}[h]
    \centering
    %\includegraphics[width=0.8\textwidth]{desfase_vs_r.pdf}
    \caption{Comparación del ángulo de desfase teórico y experimental
             en función de $\omega_f/\omega_n$.}
    \label{fig:desfase_vs_r}
\end{figure}

%----------------------------------------------------------

\subsection{Análisis de resultados}

% Aquí comentas lo que observas. Te dejo la estructura en ítems para que tú escribas.

\begin{itemize}
    \item Comparación entre $\theta^{\text{exp}}_m$ y $\theta^{\text{teo}}_m$ para
          cada configuración (abierta/cerrada). Comentar si las diferencias son
          pequeñas, moderadas o grandes.
    \item Comportamiento del FA alrededor de la resonancia
          ($r \approx 1$) y efecto del aumento de amortiguamiento
          (disminución del pico de resonancia y ensanchamiento de la curva).
    \item Comparación entre $\varphi_{\text{teo}}$ y $\varphi_{\text{exp}}$:
          tendencia del desfase desde $0^\circ$ hasta $180^\circ$ al aumentar $r$.
    \item Discusión del impacto del amortiguador (abierto vs. cerrado) en:
          \begin{itemize}
              \item amplitud máxima,
              \item posición de la frecuencia de pico,
              \item comportamiento del desfase.
          \end{itemize}
\end{itemize}

%----------------------------------------------------------

\subsubsection{Análisis de errores}

% Aquí discutes fuentes de error y su influencia en los resultados.

\begin{itemize}
    \item Incertidumbre en la medición de la frecuencia $f$ y su efecto en $r$.
    \item Errores en la calibración del LVDT (sensibilidad, posición inicial).
    \item Aproximaciones del modelo (linealidad del resorte, amortiguamiento puramente viscoso, etc.).
    \item Errores al medir $\Delta t$ para el cálculo del desfase experimental.
    \item Posibles variaciones en los parámetros físicos
          ($M_h$, $r$, momento de inercia total, etc.).
\end{itemize}

%----------------------------------------------------------

\subsection{Conclusiones específicas del experimento}

% Escribe aquí 3–5 conclusiones puntuales sobre la vibración forzada-amortiguada,
% centradas en:
% - efecto del amortiguamiento,
% - validación (o no) del modelo teórico,
% - comportamiento de FA y del desfase.

\end{document}