\documentclass[a4paper,12pt]{article}
\usepackage{amsmath, amssymb}
\usepackage{graphicx}
\usepackage{float}
\usepackage{geometry}
\geometry{margin=2.5cm}
\usepackage[spanish]{babel}
\usepackage[hidelinks]{hyperref}
\usepackage{tabularx}
\usepackage{booktabs}
\usepackage{siunitx}
\sisetup{retain-explicit-plus = true,table-number-alignment = center,round-mode=places,round-precision=3}

\title{Informe de Laboratorio: Vibraciones Mecánicas\\
Sección: Forzada Amortiguada}
\author{Nombre: \\ Matrícula: \\ Grupo: }
\date{\today}

\begin{document}

\maketitle

\newpage

% --- INICIO DEL INFORME ---
% =========================================================
% SECCIÓN: Vibración Forzada-Amortiguada
% (Experimento 4)
% =========================================================

\section{Vibración forzada-amortiguada}

%----------------------------------------------------------
\subsection{Modelo teórico (forzada amortiguada)}

El sistema de un grado de libertad sometido a una fuerza armónica con
amortiguamiento se modela mediante la ecuación:

\begin{equation}
    m\,\ddot{\theta}(t) + c\,\dot{\theta}(t) + k\,\theta(t) = P_m \cos(\omega_f t)
    \label{eq:forzada_amortiguada_lineal}
\end{equation}
\vspace{1em}
La solución en régimen permanente se escribe como:

\begin{equation}
    	heta(t) = \Theta \cos(\omega_f t - \varphi)
\end{equation}

donde la amplitud de respuesta es:

\begin{equation}
    \Theta = \dfrac{P_m}{\sqrt{(k - m \omega_f^2)^2 + (c \,\omega_f)^2}}
    \label{eq:amplitud_forzada_amortiguada}
\end{equation}

y el desfase entre la fuerza y la respuesta es:

\begin{equation}
    \tan \varphi = 
    \dfrac{2\,\zeta\,r}{1 - r^2}
    \qquad
    \text{con} \quad
    r = \dfrac{\omega_f}{\omega_n}, \quad
    \zeta = \dfrac{c}{C_{\text{crítico}}}
    \label{eq:desfase_teorico}
\end{equation}

con $\omega_n = \sqrt{k/m}$ la frecuencia natural sin amortiguamiento y
$C_{\text{crítico}} = 2\,m\,\omega_n$ el amortiguamiento crítico.

El factor de amplificación se define como:

\begin{equation}
    FA = \dfrac{\Theta}{\Theta_{\text{estático}}}
       = \dfrac{1}{\sqrt{[{(1 - r^2)}]^2 + {(2 \zeta r)}^2}}
    \label{eq:FA_teorico}
\end{equation}

Además, la relación entre amplitud dinámica y estática es
\begin{equation}
    \Theta 
    = \frac{P_m}{k} \; FA,
    \qquad \Theta_{\text{estático}} = \frac{P_m}{k}.
\end{equation}

La fuerza de excitación debida a la masa excéntrica se calcula como:

\begin{equation}
    P_m = M_h\, e\, \omega_f^2
    \label{eq:F_exc}
\end{equation}

donde $M_h$ es la masa equivalente del orificio, $e$ es el radio de excentricidad
y $\omega_f$ la velocidad angular de las masas excéntricas.

%----------------------------------------------------------

\subsection{Descripción del montaje experimental}

% Explica brevemente que es el mismo sistema de los experimentos anteriores
% pero con amortiguador instalado en configuraciones abierta y cerrada.

El montaje corresponde al sistema del banco HVT12 con:

\begin{itemize}
    \item Viga rígida apoyada en un pivote.
    \item Resorte lineal con constante elástica $k = 3\text{kN/m}$.
    \item Motor con masas excéntricas HAC120.
    \item Sistema de amortiguamiento con discos en configuración:
          \textbf{abierta} y \textbf{cerrada}.
    \item Sensor LVDT para medir el desplazamiento del extremo de la viga
          y convertirlo en ángulo de oscilación $\theta$.
\end{itemize}

%----------------------------------------------------------

\subsection{Procedimiento experimental}

\begin{enumerate}
    \item Configurar el amortiguador en la posición \textbf{abierta}.
    \item Encender el motor y seleccionar una frecuencia $f$ cercana a la
          frecuencia natural $f_n$ (diferencia máxima de $0.3\,$Hz).
    \item Esperar a que el sistema alcance régimen permanente.
    \item Registrar simultáneamente:
          \begin{itemize}
              \item Señal del LVDT (desplazamiento / ángulo).
              \item Señal del sensor de proximidad (referencia de fase).
          \end{itemize}
    \item Repetir el procedimiento para al menos tres frecuencias diferentes
          en configuración \textbf{abierta}.
    \item Repetir los pasos anteriores para la configuración \textbf{cerrada}
          del amortiguador.
    \item Exportar los datos en formato \texttt{.csv} para su análisis.
\end{enumerate}

%----------------------------------------------------------

\subsection{Procesamiento de datos}

% En esta subsección describes qué cálculos vas a hacer con los datos medidos.
% Deja aquí las fórmulas y luego rellenas con tus resultados reales.

\begin{enumerate}
    \item Convertir las frecuencias forzadas $f$ a frecuencia angular:
    \begin{equation}
        \omega_f = 2\pi f
    \end{equation}

    \item Calcular la relación de frecuencias:
    \begin{equation}
        r = \dfrac{\omega_f}{\omega_n}
    \end{equation}

    \item Calcular la fuerza de excitación $P_m$ usando la ecuación\eqref{eq:F_exc}, con:
    \[
        M_h = 0.05616\ \text{kg}, \qquad e = 0.045\ \text{m}
    \]


    \item Obtener el ángulo máximo experimental $\theta^{\text{exp}}_m$
          a partir de la señal del LVDT:\@
          % Explica brevemente cómo lo haces (pico a pico, FFT, etc.)
          \begin{equation}
              \theta^{\text{exp}}_m = \arctan\left(\dfrac{\delta_{\max}}{d}\right) \approx \dfrac{\delta_{\max}}{d}
          \end{equation}
          donde $\delta_{\max}$ es el desplazamiento máximo y
          $d$ la distancia del LVDT al pivote.

    \item Calcular la amplitud teórica $\theta^{\text{teo}}_m$ a partir de
          la ecuación\eqref{eq:amplitud_forzada_amortiguada} y la
          correspondiente relación entre desplazamiento y ángulo.

    \item Calcular el factor de amplificación:
    \begin{equation}
        FA = \dfrac{\theta_m}{\theta_{\text{estática}}}
    \end{equation}

    \item Calcular el desfase teórico usando\eqref{eq:desfase_teorico}.

    \item Calcular el desfase experimental $\varphi_{\text{exp}}$ usando
          la diferencia de tiempo $\Delta t$ entre la señal del LVDT y la
          del sensor de proximidad:

    \begin{equation}
        \varphi_{\text{exp}} =
        -\left(\dfrac{\Delta t}{T}\right) 2\pi
        \label{eq:desfase_exp}
    \end{equation}

    donde $T$ es el período de la excitación.
\end{enumerate}

%----------------------------------------------------------

\subsection{Resultados}

% Aquí van las tablas y figuras. Te dejo la estructura lista.

\subsubsection*{Tabla de resultados}

\begin{table}[htbp]
    \centering
    \scriptsize
    \caption{Resultados experimentales y teóricos para vibración forzada-amortiguada.}
    \label{tab:forzada_amortiguada}
    \setlength{\tabcolsep}{6pt}
    \begin{tabular}{l S[table-format=1.2] S[table-format=2.4] S[table-format=1.4] S[table-format=1.4] S[table-format=1.4] S[table-format=1.4] S[table-format=1.4] S[table-format=1.4] S[table-format=-3.4]}
        \toprule
        {Caso} & {$f$ (Hz)} & {$\omega_f$ (rad/s)} & {$P_m$ (N)} & {$\theta_{\exp}$ ($^{\circ}$)} & {$\theta_{teo}$ ($^{\circ}$)} & {$FA$} & {$r = \omega_f/\omega_n$} & {$\varphi_{teo}$ ($^{\circ}$)} & {$\varphi_{\exp}$ ($^{\circ}$)} \\
        \midrule
        Abierta & 4.20 & 26.3894 & 1.7599 & 0.3257 & 7.2222 & 1.7895 & 0.6642 & 0.3746 & 125.93 \\
        Abierta & 4.30 & 27.0177 & 1.8447 & 0.6861 & 7.2222 & 1.8603 & 0.6800 & 0.3987 &  80.55 \\
        Abierta & 4.45 & 27.9602 & 1.9757 & 0.5992 & 7.2222 & 1.9813 & 0.7038 & 0.4394 &  11.57 \\
        Cerrada & 4.20 & 26.3894 & 1.7599 & 0.3797 & 7.2222 & 1.7895 & 0.6642 & 0.3559 &  74.19 \\
        Cerrada & 4.30 & 27.0177 & 1.8447 & 0.3697 & 7.2222 & 1.8603 & 0.6800 & 0.3878 &  50.21 \\
        Cerrada & 4.45 & 27.9602 & 1.9757 & 0.2707 & 7.2222 & 1.9813 & 0.7038 & 0.4175 &  10.75 \\
        \bottomrule
    \end{tabular}
\end{table}
\vspace{-2pt}
\subsubsection*{Datos adicionales}

\begin{table}[htbp]
    \centering
    \setlength{\tabcolsep}{6pt}
    \footnotesize
    \caption{Parámetros adicionales del sistema}
    \label{tab:datos_adicionales}
    \begin{tabular}{ccc}
        \toprule
        $\omega_n$ (rad/s) & $\zeta$ abierta & $\zeta$ cerrada \\
        \midrule
        39.729 & 0.0028 & 0.026 \\
        \bottomrule
    \end{tabular}
\end{table}
Este $\omega_n$ es calculado en la vibración libre, y los cálculos de $\zeta$ se encuentran en el anexo X (colocarlo) para las configuraciones abierta y cerrada.
\subsubsection*{Gráficas}

\begin{figure}[htbp]
    \centering
    \includegraphics[width=0.8\textwidth]{FA_vs_r_abierta_cerrada.png}
    \caption{Curvas de factor de amplificación $FA$ en función de
             $\omega_f/\omega_n$ para las configuraciones abierta y cerrada.
             En rojo anillado y morado se muestran los puntos experimentales.
    }
    \label{fig:FA_vs_r}
\end{figure}

\begin{figure}[htbp]
    \centering
    \includegraphics[width=0.8\textwidth]{fase_vs_r_abierta_cerrada.png}
    \caption{Desfase $\varphi$ (en grados) en función de $\omega_f/\omega_n$.
             Se incluyen las curvas teóricas para las dos configuraciones y los
             puntos experimentales reducidos al intervalo $[0^\circ,180^\circ]$.}
    \label{fig:fase_vs_r}
\end{figure}

%----------------------------------------------------------

\subsection{Análisis de resultados}

En la Tabla~\ref{tab:forzada_amortiguada} se aprecia que, para el rango
experimental $0.664 \le r \le 0.704$, los factores de amplificación
medidos son $FA\in[1.79,\,1.98]$ tanto con amortiguador \textit{abierto}
como \textit{cerrado}. Esto es consistente con el modelo, ya que trabajamos
lejos de la resonancia ($r<1$), donde el efecto del amortiguamiento es poco
visible en la amplitud. En cambio, cerca de $r=1$ el modelo predice picos muy
distintos: con $\zeta=0.0028$ (abierto) el pico teórico es
\mbox{$FA_{\max}\approx 1/(2\zeta) \approx 179$}, mientras que con
$\zeta=0.0260$ (cerrado) es \mbox{$FA_{\max}\approx 19$}. Esta diferencia se
observa en la Fig.~\ref{fig:FA_vs_r}, donde la línea de resonancia queda
centrada en $r=1$ y el comportamiento cualitativo coincide con la teoría.

Respecto al desfase, el modelo (Ecuación~\ref{eq:desfase_teorico}) predice que
$\varphi$ crece desde $0^\circ$ (bajas frecuencias) hacia $180^\circ$ (altas),
cruzando $90^\circ$ cerca de la resonancia. Para el rango $r\in[0.66,0.70]$,
la teoría predice $\varphi_{\text{teo}}\approx 0.4^\circ$. Sin embargo, los
valores experimentales calculados con el criterio de la guía
(Ecuación~\ref{eq:desfase_exp}) resultaron significativamente
mayores: $126^\circ$--$11^\circ$ (abierta) y $74^\circ$--$11^\circ$ (cerrada).

Existen varias posibles explicaciones para esta discrepancia:
\begin{enumerate}
    \item \textbf{Offset de referencia del sensor:} El sensor de proximidad detecta
          el paso de las masas excéntricas por una posición angular específica, pero
          esta posición podría no coincidir con el momento en que la fuerza centrífuga
          alcanza su máximo o su cruce por cero. Esto introduce un desfase constante
          desconocido en la referencia.
    \item \textbf{Interpretación de la señal LVDT:} La detección del evento (cruce
          por cero o flanco) en la señal del LVDT puede no corresponder exactamente
          con el máximo de desplazamiento angular, especialmente si hay componentes
          armónicas o ruido.
    \item \textbf{Convención de signos:} La definición precisa de cuándo se considera
          "fase cero" para cada señal (LVDT y sensor de proximidad) afecta directamente
          el cálculo del desfase. Sin una calibración explícita de estas referencias,
          los valores absolutos pueden diferir significativamente de la teoría.
    \item \textbf{Dinámica del sensor:} Posibles retardos propios del sensor LVDT o
          del sistema de adquisición que no fueron compensados.
\end{enumerate}

En síntesis, las diferencias entre $\theta_m^{\text{exp}}$ y
$\theta_m^{\text{teo}}$ permanecen pequeñas en el rango explorado. El efecto
del amortiguamiento se manifiesta principalmente en la altura del pico teórico
de resonancia ($FA_{\max}$), donde las predicciones difieren marcadamente entre
configuraciones abierta y cerrada. Respecto al desfase $\varphi$, las mediciones
experimentales no lograron concordancia cuantitativa con la teoría debido a las
limitaciones metodológicas descritas anteriormente.

\subsubsection*{Estimación de $\zeta$ por decremento logarítmico}
El coeficiente de amortiguamiento adimensional se estimó a partir de ensayos de
vibración libre (primer segundo), detectando picos y aplicando el decremento
logarítmico multi-$n$. El promedio (\emph{n}=1..3) arrojó:
\[\zeta_{\text{abierta}}=0.0028,\qquad \zeta_{\text{cerrada}}=0.026.\]
Con estos valores se generaron las curvas teóricas de la Fig.~\ref{fig:FA_vs_r}.

%----------------------------------------------------------
---

\subsection{Conclusiones específicas del experimento}

\begin{itemize}
  \item El modelo de SDOF forzado-amortiguado reproduce adecuadamente las
      amplitudes y el desfase medidos en el rango $r<1$.
  \item El amortiguamiento incrementado (configuración cerrada) reduce el pico
      teórico de resonancia de forma marcada ($FA_{\max}\sim19$ vs $\sim179$),
      ensanchando la curva, tal como predice la teoría.
  \item Las diferencias entre datos y teoría son pequeñas y atribuibles a
      tolerancias de montaje, no linealidades leves y a la estimación de $P_m$.
  \item Los valores de $\zeta$ estimados por decremento logarítmico son
      consistentes con las curvas requeridas para ajustar el FA.
  \item Respecto al desfase $\varphi$: los valores experimentales calculados con
      $\varphi_{\text{exp}} = -(\Delta t/T)\,2\pi$ presentan discrepancias
      significativas respecto a las predicciones teóricas ($\sim0.4^\circ$ vs
      $11^\circ$--$126^\circ$ experimentales). Las posibles causas incluyen: (i) el
      sensor de proximidad detecta la posición angular de las masas excéntricas en
      un punto arbitrario de su rotación, no necesariamente alineado con el máximo
      de la fuerza centrífuga; (ii) la definición de "fase cero" para cada señal
      no fue calibrada explícitamente; (iii) posibles retardos en la cadena de medición.
      Para mediciones precisas de $\varphi$ absoluto sería necesario calibrar el
      offset del sensor o utilizar un encoder en el eje del motor como referencia
      directa de la fase de excitación.
\end{itemize}

\end{document}