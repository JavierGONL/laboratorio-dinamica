\documentclass[journal]{IEEEtran}
\renewcommand\IEEEkeywordsname{Palabras Clave}
\renewcommand{\labelitemi}{\ding{42}}
\renewcommand{\labelitemii}{\ding{43}}
\newcommand{\gt}{>}
\newcommand{\lt}{<}
\newcommand{\aprox}{\approx}
\newcommand{\gtapprox}{\gtrsim}
\newcommand{\ltapprox}{\lesssim}
\usepackage[spanish,activeacute]{babel}
\usepackage[utf8x]{inputenc}
\usepackage{graphicx}
\usepackage{booktabs}
\usepackage{amsmath}
\usepackage{cite}
\usepackage{float}
\usepackage[hidelinks]{hyperref}
\usepackage[utf8]{inputenc}
\usepackage{caption}
\usepackage[table,xcdraw]{xcolor}
\usepackage[svgnames]{xcolor}
\definecolor{darklavender}{rgb}{0.45, 0.31, 0.59}
\definecolor{brandeisblue}{rgb}{0.0, 0.44, 1.0}
\definecolor{cinnamon}{rgb}{0.82, 0.41, 0.12}
\definecolor{cardinal}{rgb}{0.77, 0.12, 0.23}

\begin{document}

\title{Analisis Experimental De Vibraciones Mecanicas}



\author{
    Hernandez Daza Juan David, Mahecha Cruz Joseph Nicolás, \\
    Maluche Suarez Ivan Felipe, Luna Gonzalez Kevin Javier \\
    \textit{Universidad Nacional de Colombia, Facultad de Ingeniería, Bogotá D.C.} \\
    \{kegonzalezl, jomahecha, imaluche, correo\}@unal.edu.co
}

\maketitle
\markboth{Universidad Nacional de Colombia, Facultad de Ingeniería, DINAMICA - 2025-2}{}

\begin{abstract}

\end{abstract}

\begin{IEEEkeywords}

\end{IEEEkeywords}

\section{Introducción}

\section{Objetivos del Proyecto}

\subsection{Objetivo General}

\subsection{Objetivos Específicos}
\begin{enumerate}

\end{enumerate}

\section{Estado del arte}


\section{Normatividad y Ensayos Técnicos(Marco Teorico)}

\subsection{Vibración Libre}

El sistema estudiado corresponde a una viga con un grado de libertad que se modela mediante la ecuación diferencial:
\begin{equation}
    I \ddot{\theta} + k \theta = 0
\end{equation}

Donde $I$ es el momento de inercia respecto al pivote, $k$ la constante elástica del resorte y $\theta$ el ángulo de oscilación. La solución de esta ecuación proporciona la frecuencia natural del sistema:
\begin{equation}
    \omega_n = \sqrt{\frac{k}{I}}
\end{equation}

A partir de la cual se obtienen el período y frecuencia natural:
\begin{equation}
    T_n = \frac{2\pi}{\omega_n}, \qquad f_n = \frac{1}{T_n}
\end{equation}

\subsection{Vibración Forzada}

% Placeholder para vibración forzada

\subsection{Vibración Libre Amortiguada}

% Placeholder para vibración libre amortiguada

\subsection{Vibración Forzada-Amortiguada}

El sistema de un grado de libertad sometido a una fuerza armónica con amortiguamiento se modela mediante la ecuación:
\begin{equation}
    m\,\ddot{\theta}(t) + c\,\dot{\theta}(t) + k\,\theta(t) = P_m \cos(\omega_f t)
\end{equation}

La solución en régimen permanente se escribe como:
\begin{equation}
    \theta(t) = \Theta \cos(\omega_f t - \varphi)
\end{equation}

donde la amplitud de respuesta es:
\begin{equation}
    \Theta = \dfrac{P_m}{\sqrt{(k - m \omega_f^2)^2 + (c \,\omega_f)^2}}
\end{equation}

y el desfase entre la fuerza y la respuesta es:
\begin{equation}
    \tan \varphi = \dfrac{2\,\zeta\,r}{1 - r^2}
    \qquad \text{con} \quad
    r = \dfrac{\omega_f}{\omega_n}, \quad
    \zeta = \dfrac{c}{C_{\text{crítico}}}
\end{equation}

con $\omega_n = \sqrt{k/m}$ la frecuencia natural sin amortiguamiento y $C_{\text{crítico}} = 2\,m\,\omega_n$ el amortiguamiento crítico.

El factor de amplificación se define como:
\begin{equation}
    FA = \dfrac{\Theta}{\Theta_{\text{estático}}} = \dfrac{1}{\sqrt{[(1 - r^2)]^2 + (2 \zeta r)^2}}
\end{equation}

La fuerza de excitación debida a la masa excéntrica se calcula como:
\begin{equation}
    P_m = M_h\, e\, \omega_f^2
\end{equation}
donde $M_h$ es la masa equivalente del orificio, $e$ es el radio de excentricidad y $\omega_f$ la velocidad angular de las masas excéntricas.

\section{Metodología Experimental}

\subsection{Vibración Libre}

El montaje corresponde al sistema del banco HVT12 con:
\begin{itemize}
    \item Viga rígida apoyada en un pivote.
    \item Resorte lineal con constante elástica $k = 3$ kN/m.
    \item Sensor LVDT para medir el desplazamiento del extremo de la viga y convertirlo en ángulo de oscilación $\theta$.
\end{itemize}

Procedimiento:
\begin{enumerate}
    \item Comprobar que tanto el motor como el amortiguador no estén conectados al sistema.
    \item Mover manualmente la viga a una posición inicial delimitada por las restricciones geométricas del sistema.
    \item Soltar para empezar la oscilación y permitir que el sistema oscile libremente.
    \item Registrar la señal del LVDT (desplazamiento/ángulo) durante varios ciclos.
    \item Exportar los datos en formato CSV para análisis.
\end{enumerate}

\subsection{Vibración Forzada}

\subsection{Vibración Forzada Amortiguada}

El montaje corresponde al sistema del banco HVT12 con:
\begin{itemize}
    \item Viga rígida apoyada en un pivote.
    \item Resorte lineal con constante elástica $k = 3$ kN/m.
    \item Motor con masas excéntricas HAC120.
    \item Sistema de amortiguamiento con discos en configuración abierta y cerrada.
    \item Sensor LVDT para medir el desplazamiento del extremo de la viga y convertirlo en ángulo de oscilación $\theta$.
\end{itemize}

Procedimiento:
\begin{enumerate}
    \item Configurar el amortiguador en la posición correspondiente (abierta o cerrada).
    \item Encender el motor y seleccionar una frecuencia $f$ cercana a la frecuencia natural $f_n$ (diferencia máxima de 0.3 Hz).
    \item Esperar a que el sistema alcance régimen permanente.
    \item Registrar simultáneamente señal del LVDT (desplazamiento/ángulo) y del sensor de proximidad (referencia de fase).
    \item Repetir para al menos tres frecuencias diferentes en cada configuración.
    \item Exportar los datos en formato CSV para análisis posterior.
\end{enumerate}

\subsubsection{Procesamiento de Datos}

\begin{enumerate}
    \item Convertir las frecuencias forzadas $f$ a frecuencia angular:
    \begin{equation}
        \omega_f = 2\pi f
    \end{equation}

    \item Calcular la relación de frecuencias:
    \begin{equation}
        r = \dfrac{\omega_f}{\omega_n}
    \end{equation}

    \item Calcular la fuerza de excitación $P_m$ usando $P_m = M_h\, e\, \omega_f^2$, con:
    \[
        M_h = 0.05616\text{ kg}, \qquad e = 0.045\text{ m}
    \]

    \item Obtener el ángulo máximo experimental $\theta^{\text{exp}}_m$ a partir de la señal del LVDT:
    \begin{equation}
        \theta^{\text{exp}}_m = \arctan\left(\dfrac{\delta_{\max}}{d}\right) \approx \dfrac{\delta_{\max}}{d}
    \end{equation}
    donde $\delta_{\max}$ es el desplazamiento máximo y $d$ la distancia del LVDT al pivote.

    \item Calcular la amplitud teórica $\theta^{\text{teo}}_m$ a partir de las ecuaciones del modelo y la correspondiente relación entre desplazamiento y ángulo.

    \item Calcular el factor de amplificación:
    \begin{equation}
        FA = \dfrac{\theta_m}{\theta_{\text{estática}}}
    \end{equation}

    \item Calcular el desfase teórico usando la ecuación del modelo.

    \item Calcular el desfase experimental $\varphi_{\text{exp}}$ usando la diferencia de tiempo $\Delta t$ entre la señal del LVDT y la del sensor de proximidad:
    \begin{equation}
        \varphi_{\text{exp}} = -\left(\dfrac{\Delta t}{T}\right) 2\pi - \dfrac{\pi}{2}
    \end{equation}
    donde $T$ es el período de la excitación.
\end{enumerate}

\subsubsection{Vibración Forzada Amortiguada cerrada}

\subsubsection{Vibración Forzada Amortiguada abierta}

\subsection{Vibración Libre Amortiguada}

\subsubsection{Vibración Libre Amortiguada cerrada}

\subsubsection{Vibración Libre Amortiguada abierta}


\section{Resultados y Análisis}

\subsection{Vibración Libre}

\subsubsection{Resultados Experimentales}

Analizando los datos obtenidos durante el experimento es posible hallar: 
\begin{itemize}
    \item $T_{prom} = 0.196$ s (período promedio de oscilación, tiempo transcurrido de cresta a cresta)
    \item $v_{max} = 500$ mV
\end{itemize}

El segundo dato nos permite obtener el valor de la distancia máximo a partir de la siguiente relación:
\[\theta_{max} = \frac{v_{max}}{350}= 1.428 \text{ mm}\]

A su vez este valor puede convertirse a un ángulo usando cinemática de cuerpo rígido: 
\[\theta_{max} = \frac{1.428}{119} = 0.01200\text{ rad}\]

\subsubsection{Valores Teóricos y Procesamiento de Datos}

Teniendo en cuenta la geometría del sistema es posible desarrollar un diagrama de cuerpo libre que permita encontrar los valores de interés:

\begin{figure}[H]
    \centering
    \includegraphics[width=0.45\textwidth]{G12/Vibracio libre/Diagrama_cuerpo_libre.png}
    \caption{Diagrama de cuerpo libre para vibración libre.}
    \label{fig:dcl_libre}
\end{figure}

Al resolver la ecuación diferencial del sistema se obtiene: 
\[\omega_{n}  = 39.67\text{ rad/s}\]

De igual manera haciendo un análisis geométrico con ayuda de los datos dados sobre el sistema es posible hallar la siguiente relación:
\[\theta_{m} = \frac{15}{845} = 0.01775\text{ rad}\]

A partir de los valores obtenidos tanto teórica como experimentalmente es posible encontrar por medio de las siguientes operaciones otros valores igual de relevantes para nuestro análisis:
\[f_{n} = \frac{1}{T_{n}}\] 
\[\omega_{n} = 2 \pi f_{n}\]

\begin{table}[H]
    \centering
    \caption{Resultados experimentales y teóricos para vibración libre.}
    \label{tab:libre_informe}
    \begin{tabular}{lccccc}
        \toprule
        Caso & $\omega_n$ [rad/s] & $T_n$ [s] & $f_n$ [Hz] & $\theta$ [rad] & $\theta$ [$^\circ$] \\
        \midrule
        EXP & 32.057 & 0.196 & 5.102 & 0.01200 & 0.6877914 \\
        TEÓRICO & 39.670 & 0.158 & 6.314 & 0.01775 & 1.016978 \\
        \bottomrule
    \end{tabular}
\end{table}

\subsubsection{Análisis}

Los resultados experimentales muestran concordancia razonable con los valores teóricos. La frecuencia natural experimental $f_n = 5.102$ Hz se aproxima al valor teórico de $6.314$ Hz, con una diferencia del orden del 19\%. Esta discrepancia puede atribuirse a:
\begin{itemize}
    \item Tolerancias en las propiedades geométricas e inerciales del sistema.
    \item Fricción en el pivote no considerada en el modelo ideal.
    \item Incertidumbre en la constante elástica del resorte.
\end{itemize}

El período promedio medido $T_{prom} = 0.196$ s proporciona una base sólida para los análisis de vibración forzada subsecuentes, estableciendo la frecuencia de referencia del sistema.

\subsubsection{Gráfica ángulo vs tiempo}

En la Figura~\ref{fig:libre_angulo_tiempo_informe} se presenta la gráfica obtenida a partir de los datos del archivo CSV, que describe cómo varía la señal eléctrica del sensor en función del tiempo:

\begin{figure}[H]
    \centering
    \includegraphics[width=0.48\textwidth]{G12/Vibracio libre/todos_B.png}
    \caption{Ángulo vs tiempo para vibración libre.}
    \label{fig:libre_angulo_tiempo_informe}
\end{figure}

\subsection{Vibración Forzada Amortiguada}

\subsubsection{Resultados Experimentales}

En las Tablas~\ref{tab:forzada_parametros} y~\ref{tab:forzada_resultados} se presentan los parámetros de entrada y resultados calculados para vibración forzada-amortiguada.

\begin{table}[H]
    \centering
    \caption{Parámetros de entrada y amplitudes medidas}
    \label{tab:forzada_parametros}
    \scriptsize
    \begin{tabular}{lccccc}
        \toprule
        Caso & $f$ (Hz) & $\omega_f$ (rad/s) & $P_m$ (N) & $\theta_{\exp}$ ($^\circ$) & $\theta_{teo}$ ($^\circ$) \\
        \midrule
        Abierta & 4.20 & 26.3894 & 1.7599 & 0.3257 & 7.2222 \\
        Abierta & 4.30 & 27.0177 & 1.8447 & 0.6861 & 7.2222 \\
        Abierta & 4.45 & 27.9602 & 1.9757 & 0.5992 & 7.2222 \\
        Cerrada & 4.20 & 26.3894 & 1.7599 & 0.3797 & 7.2222 \\
        Cerrada & 4.30 & 27.0177 & 1.8447 & 0.3697 & 7.2222 \\
        Cerrada & 4.45 & 27.9602 & 1.9757 & 0.2707 & 7.2222 \\
        \bottomrule
    \end{tabular}
\end{table}

\begin{table}[H]
    \centering
    \caption{Resultados calculados: FA, r y desfases}
    \label{tab:forzada_resultados}
    \scriptsize
    \begin{tabular}{lcccc}
        \toprule
        Caso & $FA$ & $r$ & $\varphi_{teo}$ ($^\circ$) & $\varphi_{\exp}$ ($^\circ$) \\
        \midrule
        Abierta & 1.7895 & 0.6642 & 0.3746 & 125.93 \\
        Abierta & 1.8603 & 0.6800 & 0.3987 &  80.55 \\
        Abierta & 1.9813 & 0.7038 & 0.4394 &  11.57 \\
        Cerrada & 1.7895 & 0.6642 & 0.3559 &  74.19 \\
        Cerrada & 1.8603 & 0.6800 & 0.3878 &  50.21 \\
        Cerrada & 1.9813 & 0.7038 & 0.4175 &  10.75 \\
        \bottomrule
    \end{tabular}
\end{table}

\begin{table}[H]
    \centering
    \caption{Parámetros adicionales del sistema}
    \label{tab:datos_adicionales_informe}
    \begin{tabular}{ccc}
        \toprule
        $\omega_n$ (rad/s) & $\zeta$ abierta & $\zeta$ cerrada \\
        \midrule
        39.729 & 0.0028 & 0.026 \\
        \bottomrule
    \end{tabular}
\end{table}

Este $\omega_n$ es calculado en la vibración libre, y los cálculos de $\zeta$ se obtuvieron por decremento logarítmico para las configuraciones abierta y cerrada.

\subsubsection{Análisis del Factor de Amplificación}

Los factores de amplificación medidos son $FA \in [1.79, 1.98]$ para ambas configuraciones. Esto es consistente con el modelo, ya que trabajamos lejos de la resonancia ($r < 1$), donde el efecto del amortiguamiento es poco visible en la amplitud. 

En cambio, cerca de $r = 1$ el modelo predice picos muy distintos: con $\zeta = 0.0028$ (abierto) el pico teórico es $FA_{max} \approx 1/(2\zeta) \approx 179$, mientras que con $\zeta = 0.026$ (cerrado) es $FA_{max} \approx 19$. Esta diferencia se observa en la Fig.~\ref{fig:FA_vs_r_informe}, donde la línea de resonancia queda centrada en $r = 1$ y el comportamiento cualitativo coincide con la teoría.

\begin{figure}[H]
    \centering
    \includegraphics[width=0.48\textwidth]{FA_vs_r_abierta_cerrada.png}
    \caption{Factor de amplificación $FA$ vs $r = \omega_f/\omega_n$}
    \label{fig:FA_vs_r_informe}
\end{figure}

\subsubsection{Análisis del Desfase}

En la Figura~\ref{fig:fase_vs_r_informe} se presenta la gráfica del desfase en función de la relación de frecuencias:

\begin{figure}[H]
    \centering
    \includegraphics[width=0.48\textwidth]{fase_vs_r_abierta_cerrada.png}
    \caption{Desfase $\varphi$ (en grados) en función de $\omega_f/\omega_n$. Se incluyen las curvas teóricas para las dos configuraciones y los puntos experimentales.}
    \label{fig:fase_vs_r_informe}
\end{figure}

El modelo teórico predice que $\varphi$ crece desde $0^\circ$ (bajas frecuencias) hacia $180^\circ$ (altas), cruzando $90^\circ$ cerca de la resonancia. Para el rango $r \in [0.66, 0.70]$, la teoría predice $\varphi_{teo} \approx 0.4^\circ$. Sin embargo, los valores experimentales resultaron significativamente mayores: $126^\circ$--$11^\circ$ (abierta) y $74^\circ$--$11^\circ$ (cerrada).

Posibles explicaciones para esta discrepancia:
\begin{enumerate}
    \item \textbf{Offset de referencia del sensor:} El sensor de proximidad detecta el paso de las masas excéntricas por una posición angular específica que podría no coincidir con el momento en que la fuerza centrífuga alcanza su máximo o cruce por cero.
    \item \textbf{Convención de signos:} La definición de "fase cero" para cada señal afecta directamente el cálculo. Sin calibración explícita, los valores absolutos pueden diferir significativamente de la teoría.
    \item \textbf{Dinámica del sensor:} Posibles retardos propios del sensor LVDT o del sistema de adquisición no compensados.
\end{enumerate}

Las mediciones experimentales del desfase no lograron concordancia cuantitativa con la teoría debido a estas limitaciones metodológicas.

\section{Discusiones}


\section{Conclusiones}

\subsection{Vibración Libre}

\begin{enumerate}
    \item Se determinó experimentalmente la frecuencia natural del sistema: $f_n = 5.102$ Hz ($\omega_n = 32.057$ rad/s), con un período promedio de $T_{prom} = 0.196$ s.
    \item Los valores experimentales muestran concordancia cualitativa con las predicciones teóricas ($f_n^{teo} = 6.314$ Hz), con diferencias atribuibles a tolerancias geométricas, fricción en el pivote e incertidumbre en parámetros del sistema.
    \item El ángulo máximo de oscilación experimental $\theta_{max} = 0.012$ rad ($0.688^\circ$) proporciona información sobre las condiciones iniciales del sistema.
    \item Los valores obtenidos en este experimento establecen la base de referencia para caracterizar el comportamiento del sistema en condiciones de vibración forzada y amortiguada.
\end{enumerate}

\subsection{Vibración Forzada Amortiguada}

\begin{enumerate}
    \item El modelo de SDOF forzado-amortiguado reproduce adecuadamente las amplitudes medidas en el rango $r < 1$.
    \item El amortiguamiento incrementado (configuración cerrada) reduce el pico teórico de resonancia de forma marcada ($FA_{max} \sim 19$ vs $\sim 179$), ensanchando la curva, tal como predice la teoría.
    \item Las diferencias entre $\theta_{exp}$ y $\theta_{teo}$ permanecen pequeñas en el rango explorado. El efecto del amortiguamiento se manifiesta principalmente en la altura del pico teórico de resonancia.
    \item Los valores de $\zeta$ estimados por decremento logarítmico ($\zeta_{abierta} = 0.0028$, $\zeta_{cerrada} = 0.026$) son consistentes con las curvas teóricas requeridas para ajustar el factor de amplificación.
    \item Respecto al desfase $\varphi$: los valores experimentales presentan discrepancias significativas respecto a las predicciones teóricas ($\sim 0.4^\circ$ vs $11^\circ$--$126^\circ$ experimentales). Las posibles causas incluyen: (i) offset de referencia del sensor de proximidad que detecta una posición angular arbitraria no alineada con la fuerza centrífuga; (ii) la definición de ``fase cero'' no fue calibrada explícitamente; (iii) posibles retardos en la cadena de medición. Para mediciones precisas de $\varphi$ absoluto sería necesario calibrar el offset del sensor o utilizar un encoder en el eje del motor como referencia directa de la fase de excitación.
\end{enumerate}

\bibliography{bibliografia}
\bibliographystyle{IEEEtran}

\section{Anexos}

\end{document}
