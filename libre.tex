\documentclass[12pt]{article}

% Paquetes
\usepackage{amsmath, amssymb}
\usepackage{graphicx}
\usepackage{booktabs}
\usepackage{siunitx}
\usepackage{caption}
\usepackage{geometry}
\geometry{margin=2.5cm}

\begin{document}
\section{Vibración libre}
En esta seccion se abordaran los resultados obtenidos en el primer experimento, correspondiente a la vibracion libre de la viga. Esta seccion dara las bases para el desarrollo de todos lso demas casos de vibraciones

\subsection{Calculo de Valores experimentales}
A partir de los datos obtenidos durante el experimento adjuntos en los archivos CSV se obtuvieron los siguientes valores: 
\[T_{prom} = 0.196\ s\]
\[\theta_{m} = 0.012\ rad\]
Estos valores para el periodo y amplitud maxima angular son contrastados con sus equivalentes teoricos, los cuales a base de los datos fisicos del sistema (masa, constante de resorte, etc)   por medio de los procesos adjuntos se pueden obtener:
\[\omega_{n}  = 39.67\ rad/s\]
\[\theta_{m} = 0.01775\ rad\]
\subsection{Procesamiento de datos}
A partir de estos 2 valores es posible encontrar por medio de las siguientes operaciones otros valores igual de relevantes para nuestro analisis:
\noindent \textbf{Frecuencia natural}:
\[f_{n} = \frac{1}{T_{prom}} = 5.10\ Hz\] 
\[w_{n} = 2 \pi f_{n} = 32.04\ rad/s\]
\subsection{Cuadro de resultados}

\begin{table}[h!]
\centering
\caption{Resultados experimentales y teóricos para vibración libre.}

\label{tab:libre_cuadro}
\begin{tabular}{lccccc}
	\toprule
Caso & $\omega_n$ [rad/s] & $T_n$ [s] & $f_n$ [Hz] & $\theta$ [rad] & $\theta$ [°] \\
\midrule
EXP & 32.057 & 0.196 & 5.102 & 0.01200 & 0.6877914 \\
TEÓRICO & 39.670 & 0.158 & 6.314 & 0.01775 & 1.016978 \\
\bottomrule
\end{tabular}
\end{table}

\subsection{Gráfica ángulo vs tiempo}

En la Figura \ref{fig:libre_angulo_tiempo} se presenta la gráfica del ángulo en función del tiempo para la vibración libre de la viga, obtenida a partir de los datos experimentales del sensor de proximidad (Canal B).

\begin{figure}[h!]
\centering
\includegraphics[width=0.8\textwidth]{C:/Users/ivanm/OneDrive/Documentos/Dinamica/lab_dinamica/laboratorio-dinamica/G12/Vibracio libre/todos_B.png}
\caption{Señal vs tiempo para vibración libre.}
\label{fig:libre_angulo_tiempo}
\end{figure}

\subsection{Conclusiones}
A partir de los resultados obtenidos en este experimento es posible observar que los valores teoricos no estan muy alejados de los medidos mediante el experimento, lo cual indica que la metodologia empleada es adecuada para el analisis de este tipo de sistemas. Sin embargo, se observa una diferencia notable en los valores de frecuencia natural y periodo, lo cual puede atribuirse a factores como errores en la medicion, condiciones ambientales o propiedades del material que no fueron consideradas en el modelo teorico. Estos resultados seran de utilidad para complementar siguientes secciones

\end{document}