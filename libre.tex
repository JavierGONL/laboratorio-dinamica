\documentclass[12pt]{article}

% Paquetes
\usepackage{amsmath, amssymb}
\usepackage{graphicx}
\usepackage{booktabs}
\usepackage{siunitx}
\usepackage{caption}
\usepackage{geometry}
\geometry{margin=2.5cm}

\begin{document}
\section{Vibración libre}
En esta seccion se abordaran los resultados obtenidos en el primer experimento correspondiente a la vibracion libre de la viga. Esta seccion dara las bases para el desarrollo de todos los demas casos de vibraciones vistos en el laboratorio.

\subsection{Calculo de Valores experimentales}
Analizando las graficas es posible hallar los siguientes valores: partir de los datos obtenidos durante el experimento y por medio de un alaisis posterior a los archivos csv fue posible obtener: 
\[T_{prom} = 0.196\ s\] (periodo promedio de oscilacion, tiempo transcurrido de cresta a cresta)
\[\v_{max} = 500\ mv \]
El segundo dato nos permite obtener el valor de la distancia maximo a paritr de la siguiente relacion 
\[\theta_{max} = \frac{v_{max}}{350}= 1.428 mm\]
A su vez este valro puede convertirse a un angulo usando cinematica de cuerpo rigido: 
\[\theta_{max} = \frac{1.428}{119} = 0.01200\ rad\]
\subsection{Valores teoricos y Procesamiento de datos}
Teniendo en cuenta la geometria del sistema es posible desarrollar un diagrama de cuerpo libre que permita encontrar los valores de interes:
\begin{figure}[h!]
\centering
\includegraphics[width=0.5\textwidth]{C:/Users/ivanm/OneDrive/Documentos/Dinamica/lab_dinamica/laboratorio-dinamica/G12/Vibracio libre/Diagrama_cuerpo_libre.png}
\caption{Diagrama de cuerpo libre para vibración libre.}
\end{figure}
Al resolver la ecuacion diferencial del sistema se obtiene: 
\[\omega_{n}  = 39.67\ rad/s\]
De igual manera haciendo un analisis geometrico con ayuda de los datos dados sobre el sistema es psoible hallar la siguiente relacion:
\[\theta_{m} = \frac{15}{845}\]
\[\theta_{m} = 0.01775\ rad\]
\subsection{Procesamiento de datos}
A partir de los valores obtenidos tanto teorica como experimentalmente es posible encontrar por medio de las siguientes operaciones otros valores igual de relevantes para nuestro analisis:
\[f_{n} = \frac{1}{T_{n}} \] 
\[w_{n} = 2 \pi f_{n} \]

\begin{table}[h!]
\centering
\caption{Resultados experimentales y teóricos para vibración libre.}

\label{tab:libre_cuadro}
\begin{tabular}{lccccc}
	\toprule
Caso & $\omega_n$ [rad/s] & $T_n$ [s] & $f_n$ [Hz] & $\theta$ [rad] & $\theta$ [°] \\
\midrule
EXP & 32.057 & 0.196 & 5.102 & 0.01200 & 0.6877914 \\
TEÓRICO & 39.670 & 0.158 & 6.314 & 0.01775 & 1.016978 \\
\bottomrule
\end{tabular}
\end{table}

\subsection{Gráfica ángulo vs tiempo}

En la Figura \ref{fig:libre_angulo_tiempo} se presenta la gráfica obtenida a partir de los datos dados por el archivo csv, esta describe el como varia la señal electrica del sensor en funcion del tiempo:

\begin{figure}[h!]
\centering
\includegraphics[width=0.8\textwidth]{C:/Users/ivanm/OneDrive/Documentos/Dinamica/lab_dinamica/laboratorio-dinamica/G12/Vibracio libre/todos_B.png}
\caption{Señal vs tiempo para vibración libre.}
\label{fig:libre_angulo_tiempo}
\end{figure}

\subsection{Conclusiones}
A partir de los resultados obtenidos en este experimento es posible observar concordancia con los hipoteticos valores teoricos, lo cual indica que la metodologia empleada es adecuada para el analisis de este tipo de sistemas. Sin embargo, se observa que esta diferencia aun existe. Esto se puede acuñar a el desprecio de otros valores importantes que afectan el movimiento del sistema (resistencia del aire, dilatancion de la viga por el calor, etc). Estos resultados seran de utilidad para complementar siguientes secciones

\end{document}