\documentclass[12pt]{article}

% Paquetes
\usepackage{amsmath, amssymb}
\usepackage{graphicx}
\usepackage{booktabs}
\usepackage{siunitx}
\usepackage{caption}
\usepackage{geometry}
\geometry{margin=2.5cm}

\begin{document}

\section{Vibración Forzada-Amortiguada (Configuración Abierta)}

En esta sección se presentan los resultados experimentales obtenidos para la condición 
de \textbf{amortiguador abierto} con una frecuencia de excitación aproximada de 
\(\,f \approx \SI{4.18}{Hz}\). Los valores experimentales fueron calculados a partir de 
las señales registradas del LVDT (Canal A) y el sensor de proximidad (Canal B) 
proporcionados por el sistema de adquisición de datos.

\subsection{Cálculo de valores experimentales}

A partir de los seis archivos CSV suministrados, se obtuvo para cada corrida la amplitud
del LVDT, el período \(T\), la frecuencia forzada \(f\) y el desfase 
experimental \(\varphi^{exp}\). 

Los valores promediados fueron:

\[
A_{\text{LVDT,prom}} = 5.31\ \text{V},\qquad
T_{\text{prom}} = 0.239\ \text{s},\qquad
f_{\text{prom}} = 4.18\ \text{Hz},
\]
\[
\varphi^{exp}_{\text{prom}} = -55.8^\circ.
\]

\subsection{Ecuaciones relevantes}

\noindent \textbf{Relación de frecuencias}:
\[
r = \frac{\omega_f}{\omega_n} = 
\frac{2\pi f}{2\pi f_n} = \frac{f}{f_n}.
\]

\noindent \textbf{Amplitud angular experimental}:
\[
X_m^{exp} = A_{\text{LVDT}}\, S_{\text{LVDT}}, \qquad
\theta_m^{exp} = \frac{X_m^{exp}}{L}.
\]

\noindent \textbf{Factor de amplificación teórico}:
\[
FA(r) = \frac{1}{\sqrt{(1-r^2)^2 + (2\zeta r)^2}}.
\]

\noindent \textbf{Desfase teórico}:
\[
\tan\varphi^{teo} = \frac{2\zeta r}{1 - r^2}.
\]

\noindent \textbf{Desfase experimental} (según la guía):
\[
\varphi^{exp} = 
-\,\frac{\Delta t}{T} \, 2\pi - \frac{\pi}{2}.
\]


\subsection{Cuadro de resultados (Configuración Abierta, 4.18 Hz)}

En la Tabla \ref{tab:cuadroX} se presenta el resumen de los resultados experimentales
junto con los espacios destinados a los valores que dependen de la calibración del LVDT,
la frecuencia natural \(f_n\), el amortiguamiento \(\zeta\) y la fuerza excitadora \(P_m\).

\begin{table}[h!]
\centering
\caption{Resultados experimentales y teóricos para la condición Abierta (4.18 Hz).}
\label{tab:cuadroX}
\begin{tabular}{lccccccccc}
\toprule
Caso & \(f\) [Hz] & \(P_m\) [N] & \(\theta_m^{exp}\) [°] &
\(\theta_m^{teo}\) [°] & FA & \(r\) &
\(\varphi^{teo}\) [°] & \(\varphi^{exp}\) [°] \\
\midrule
Abierta & 4.18 & \_\_\_ & \_\_\_ & \_\_\_ & \_\_\_ & \_\_\_ &
\_\_\_ & -55.8 \\
\bottomrule
\end{tabular}
\end{table}

\subsection{Gráfica del factor de amplificación}

La gráfica solicitada por la guía consiste en representar el factor de amplificación FA
contra la razón de frecuencias \(r = \omega_f / \omega_n\), incluyendo:

\begin{itemize}
    \item Curva teórica del FA para la condición Abierta.
    \item Punto experimental correspondiente a \(f = 4.18\ \text{Hz}\).
\end{itemize}

Una vez se conozca \(f_n\) y \(\zeta\), la curva teórica se obtiene a partir de:

\[
FA(r) = \frac{1}{\sqrt{(1-r^2)^2 + (2\zeta r)^2}}.
\]

El punto experimental se coloca en:
\[
(r_{exp},\, FA_{exp}) =
\left(\frac{f}{f_n},\ \frac{\theta_m^{exp}}{\theta_{estática}}\right).
\]

\section{Conclusiones}

\begin{itemize}
    \item La frecuencia forzada promedio obtenida del sistema fue 
    \(f = \SI{4.18}{Hz}\), consistente con el valor programado.
    \item El desfase experimental promedio encontrado fue 
    \(\varphi^{exp} \approx -55.8^\circ\).
    \item Los valores teóricos dependen de \(f_n\), \(\zeta\) y la calibración del LVDT,
    por lo que deben completarse al finalizar los cálculos correspondientes.
\end{itemize}

\end{document}
